\documentclass[spanish,notitlepage,letterpaper, 12pt]{article} 
\usepackage[spanish]{babel} 
\usepackage{amsmath}
\usepackage{amsfonts}
\usepackage{amssymb}
\usepackage{graphicx}
\usepackage{geometry}      
\geometry{letterpaper}                  
\usepackage{epstopdf}
\usepackage{fancyhdr} % Paquete para encabezados y pies de pag
\usepackage{color}
\usepackage{placeins}
\usepackage{csquotes}
\usepackage{textcomp}
\usepackage{gensymb} 
\usepackage{tikz}

\graphicspath{images/}

\pagestyle{fancy} 
\chead{\bfseries Informe 5 - Grupo C1B. Subgrupo 2} 
\rhead{10 de Junio de 2023}
\cfoot{Universidad Industrial de Santander} 
\rfoot{\thepage} 

\voffset = -0.25in 
\textheight = 8.0in 
\textwidth = 6.5in
\oddsidemargin = 0.in
\headheight = 20pt 
\headwidth = 6.5in
\renewcommand{\headrulewidth}{0.5pt}
\renewcommand{\footrulewidth}{0,5pt}
\begin{document}
\begin{titlepage}
    \begin{center}
        \includegraphics[width=0.4\textwidth]{../general-images/uis-logo.png}
        
        \vspace{0.5cm}
        \LARGE
        \textbf{Estudio de la amplitud de las oscilaciones armónicas amortiguadas y forzadas}
        
        \vspace{0.5cm}
        \large
        Informe 3
        
        \vfill
        
        \textbf{Daniel Esteban Vargas Reyes.} Estudiante - Geología\\
        \textbf{Nicolás Andrés Ramírez Calderón.} Estudiante - Ingeniería de Sistemas\\ 
        \textbf{Rodolfo Valentín Muñoz Vega.} Estudiante - Ingeniería Química\\

        \vspace{1.0cm}
        Presentado a la docente:
        
        \textbf{Zayda Paola Reyes Quijano}
        
        \vfill
        
        Escuela de Física - Física III\\
        Universidad Industrial de Santander\\
        Bucaramanga, Santander, Colombia\\
        05 de Mayo de 2023        
    \end{center}
\end{titlepage}

\tableofcontents

\newpage

\section{Resumen}
El sonido y la luz son ejemplos de ondas aparentemente muy distintas, pero con
muchas características en común. Por ejemplo, ambas se propagan en tres dimensiones y
tienen propiedades como la reflexión y la refracción. Las ondas ultrasónicas son ondas
mecánicas de frecuencias mayores a 20 KHz que pueden propagarse prácticamente en
cualquier material e incluso el organismo, sin lesionar los tejidos. Esto ha hecho posible la
aplicación de las ondas ultrasónicas en áreas como la medicina, en la formación de
imágenes diagnósticas, en el campo industrial, entre otros. Estas ondas pueden ser generadas
gracias al efecto piezoeléctrico al someter un cristal a un campo eléctrico, produciendo
vibraciones mecánicas de cierta frecuencia que se convierten en la onda de ultrasonido.\par
\bigskip
Otra de las aplicaciones interesantes del sonido y el ultrasonido es la determinación de una
distancia, por ejemplo, entre el fondo del lecho submarino y una embarcación usando el
principio de la ecosonda. Este principio usa una velocidad del sonido conocida donde el
tiempo que invierte la onda desde que es emitida hasta que es recibida en el mismo punto
de emisión, permite determinar la distancia entre el emisor y la superficie de reflexión de la
onda.\par
\bigskip
En el presente proyecto se estudió el fenómeno de la reflexión en las ondas ultrasónicas
con el fin de comprobar esta ley. Por otra parte, con la ayuda del principio de la ecosonda
se determinó la velocidad de la onda ultrasónica.
\section{Introducción}
En este proyecto de investigación se busca estudiar la generación, propagación y
reflexión de las ondas mecánicas de frecuencia de 40 KHz, conocidas como ondas
ultrasónicas. Por otra parte, se estudió el principio de ecosonda y a partir de las distancias
entre el emisor-receptor y el obstáculo, se determinó la velocidad del sonido. Esto genera
la pregunta: En el fenómeno de la reflexión de la onda ultrasónica, ¿Cómo es la relación entre el ángulo de incidencia y el ángulo de reflexión? ¿A partir del principio de ecosonda
es posible encontrar el valor de la velocidad del sonido y las distancias entre el emisor y el
obstáculo que genera la reflexión de la onda?
\subsection{Marco teórico} \label{I.MT}
Los materiales piezoeléctricos pueden convertir una tensión mecánica en una señal
eléctrica y una señal eléctrica en vibraciones mecánicas. Por ejemplo, cuando un cristal de
cuarzo se somete a una presión fuerte, el orden de los átomos que lo componen se altera
ligeramente, generando una diferencia de potencial. Por el contrario, el cristal en presencia
de un campo eléctrico oscilante puede alterar su forma, generando de esta manera
vibraciones mecánicas que al coincidir con la frecuencia natural del cristal, serán de notable
amplitud como consecuencia del fenómeno de resonancia.\par
\bigskip
Se usan dos transductores de
ultrasonido de aproximadamente $40\ KHz$, uno receptor y el otro emisor. El emisor se
considera como una fuente ultrasónica puntual y se coloca en el punto focal de un reflector cóncavo con el fin de formar una onda plana. Cuando una onda plana se propaga en un
medio (aire) e incide en una superficie que divide dos medios, se forma una onda plana
reflejada. Se puede determinar experimentalmente que los frentes de onda reflejados
forman el mismo ángulo con la superficie que los frentes de onda incidentes. El rayo
incidente y la normal a la superficie forman el plano de incidencia. La ley de la reflexión
describe la relación entre esos ángulos: \textit{El rayo reflejado está en el plano de incidencia. El
ángulo de incidencia es igual al ángulo de reflexión.}\par
\shorthandoff{>}
\begin{figure}[!ht]
    \centering
    \begin{tikzpicture}%[scale=1.5, transform shape]
        \coordinate (hp) at (0,0);

        %Reflecting screen
        \draw [very thick] (0,2.5) -- (0,-2.5);

        %Direction of the sound
        \draw [->,thick] (-3,2) -- (hp);
        \draw [->,thick] (hp) -- (-3,-2);

        %Sound waves
        %Incident

        %first
        \draw (-2.6,1.73) arc [start angle=-33.69, end angle=-20, radius=1];
        \draw (-2.6,1.73) arc [start angle=-33.69, end angle=-47.38, radius=1];
        %second
        \draw (-2.5,1.66) arc [start angle=-33.69, end angle=-15, radius=1];
        \draw (-2.5,1.66) arc [start angle=-33.69, end angle=-52.38, radius=1];
        %third
        \draw (-2.4,1.6) arc [start angle=-33.69, end angle=-10, radius=1];
        \draw (-2.4,1.6) arc [start angle=-33.69, end angle=-57.38, radius=1];

        %Reflected

        %third
        \draw (-2.6,-1.73) arc [start angle=213.69, end angle=190, radius=1];
        \draw (-2.6,-1.73) arc [start angle=213.69, end angle=237.38, radius=1];
        %second
        \draw (-2.5,-1.66) arc [start angle=213.69, end angle=195, radius=1];
        \draw (-2.5,-1.66) arc [start angle=213.69, end angle=232.38, radius=1];
        %first
        \draw (-2.4,-1.6) arc [start angle=213.69, end angle=200, radius=1];
        \draw (-2.4,-1.6) arc [start angle=213.69, end angle=227.38, radius=1];

        %Dashed line to distinguish the angles
        \draw [dashed] (-2.5,0) -- (2.5,0);

        %Angles themselves
        \draw [red] (-1,0) arc [start angle=180, end angle=146.3099, radius=1];
        \draw [blue] (-1,0) arc [start angle=180, end angle=213.69, radius=1];

        %Labels
        \draw [red, align=left] (163.15:2.2) node {\footnotesize Ángulo de\\\footnotesize incidencia};
        \draw [align=left] (-30:1) node {\footnotesize Punto de\\\footnotesize incidencia};

        % Grid
        %\draw[help lines] (-3,-3) grid (3,3);
    \end{tikzpicture}
    \caption{\textit{Ley de la reflexión.}}
    \label{fig:reflection-law}
\end{figure}
\shorthandon{>}
Los ultrasonidos se reflejan en las superficies que dividen dos medios de diferente
impedancia acústica y densidad. Un transductor configurado para operar como emisor,
emite señales de impulsos ultrasónicos y el sistema de ecosonda mide el tiempo de tránsito
del pulso desde el emisor hasta el receptor después de reflejarse en la superficie. Para
simplificar la configuración, el emisor y el receptor están en el mismo lugar y el tiempo entre
la transmisión y la recepción se puede utilizar para determinar la distancia del transmisor a
la superficie reflectante (si la velocidad del sonido es conocida \eqref{eq:speed-sound}) o también para determinar
la velocidad del sonido si la distancia recorrida por el pulso es conocida.
\begin{equation}\label{eq:speed-sound}
    v=331\sqrt{1+\frac{T_c}{273}}
\end{equation}
Para los gases, la velocidad del sonido está dada por el módulo de volumen $B$ y la densidad volumétrica en cuestión $\rho$. Además, la velocidad del sondio \eqref{eq:speed-sound} también depende de la temperatura del medio. \cite{serway_jewett_2017}
\section{Metodología}
El desarrollo del proyecto se realizó en tres etapas o fases metodológicas, en las cuales
se realizó la recolección de los datos y posteriormente su respectivo procesamiento. En la
primera fase, se observó la generación de ondas ultrasónicas y luego su reflexión en una
superficie metálica con el objetivo de verificar la ley de la reflexión. En la segunda fase, se
verificó el principio de ecosonda para posteriormente encontrar el valor experimental del
sonido y la distancia entre dos obstáculos. Finalmente, en la fase tres se procesó la
información obtenida en las primeras fases con el objetivo de demostrar la ley de la reflexión
y de obtener la velocidad del sonido.
\subsection{Materiales}
\begin{itemize}
    \item Dos transductores ultrasónicos de $40\ KHz$.
    \item Generador de $40\ Khz$.
    \item Amplificador de AC.
    \item Un espejo cóncavo.
    \item Un osciloscopio de dos canales.
    \item Una base de pieza giratoria con escala angular.
    \item Una placa de reflexión.
\end{itemize}
\subsection{Fase 1}
En esta primera fase se observó la generación de las ondas de ultrasonido y
se verificó la ley de la reflexión. Para esto, se contó con un transductor como transmisor
ubicado en el punto focal de un reflector cóncavo, con el fin de formar una onda plana
propagada en el aire hasta una superficie reflectora. La onda reflejada fue captada por un
transductor en modo receptor, el cual se encontraba acoplado a un osciloscopio que registraba la
onda de voltaje, cuya amplitud es proporcional a la amplitud de la onda de ultrasonido
captada. El emisor se colocó a un ángulo fijo $\alpha=45\degree$ entre la normal y el transductor
emisor (Fig. \ref{fig:reflection-law}). Luego se ajustó el ángulo $\beta$ en $45\degree$ entre el receptor y la normal de
la superficie reflectora. Se registró la amplitud de la onda de voltaje mostrada en el
osciloscopio. Este procedimiento se repitió para diferentes ángulos $\beta$,
haciendo un barrido paso a paso de $3\degree$ para ángulos mayores y menores de $45\degree$,
conservando el ángulo $\alpha$. Luego, se ajustó un nuevo ángulo $\alpha$ y nuevamente se realizó
un barrido para registrar el ángulo $\beta$ que correspondía a la máxima intensidad
captada por el receptor, es decir, el ángulo de reflexión.
\subsection{Fase 2}
En esta fase se verificó el principio de la ecosonda con el fin de obtener la
velocidad del sonido y la distancia entre dos obstáculos. Para esto, se realizó el montaje
indicado (Fig. \ref{fig:ecosound}). Primero, se dispuso de los dos transductores piezoeléctricos,
emisor y receptor (juntos) y a la misma distancia $d$ de la placa reflectora. Luego, se registró el tiempo del pulso sonoro de acuerdo
a la señal en el osciloscopio. Este tiempo es el que invierte la onda en recorrer
la distancia $2d$, es decir, desde que es emitida hasta que es recibida por el receptor. La
distancia $d$ se variaró entre $10\ cm$ y $1m$ para registrar los datos de tiempo contra distancia. Posteriormente, para
dos distancias $d$ se colocó un segundo obstáculo y se registraron los valores del tiempo del
pulso sonoro entre los dos obstáculos. Por último, se registró el valor de la
temperatura del laboratorio.
\shorthandoff{>}
\begin{figure}[!ht]
    \centering
    \begin{tikzpicture}%[scale=1.5, transform shape]
        %Reflector
        \draw [very thick] (2,1) -- (2,-1);

        %Distance
        \draw [->,thick] (0.5,-0.2) -- (2,-0.2);
        \draw [->,thick] (-0.5,-0.2) -- (-2,-0.2);
        \draw [dashed] (-2, 0.2) -- (2, 0.2);

        %Transductors
        %first
        \draw (-2,0.3) -- ++(0,0.5) -- ++(-1,0) -- ++(0,-0.5) -- ++(1,0);
        \draw (-2,0.1) -- ++(0,-0.5) -- ++(-1,0) -- ++(0,0.5) -- ++(1,0);

        %Labels
        \draw [align=left] (35:1.5) node {\footnotesize Reflector};
        \draw [align=left] (158.198:2.69) node {\footnotesize Transductores};
        \draw [align=left] (-90:0.25) node {\footnotesize $d$};

        % Grid
        %\draw[help lines] (-3,-3) grid (3,3);
    \end{tikzpicture}
    \caption{\textit{Montaje ecosonda.}}
    \label{fig:ecosound}
\end{figure}
\shorthandon{>}

\subsection{Fase 3}
Se procesó la información obtenida en las primeras fases con el objetivo de
demostrar la ley de la reflexión, verificar el principio de la ecosonda y obtener la velocidad
del sonido. Primero, se analizó la información registrada en las tablas de datos. Luego,
se realizó la verificación de la ley de la reflexión, de acuerdo a los datos registrados, se graficó el voltaje $V$ que indica la intensidad reflejada (ordenadas) en función
del ángulo $\beta$ (abscisas). Después se procesió a realizar un ajuste a los datos mediante
una gaussiana, se observó el perfil obtenido y se hizo un análisis del resultado. Luego,
se determinó la anchura angular del valor medio de la distribución de intensidad.\par
\bigskip
Segundo, se construyó una gráfica de $\beta$ en función de $\alpha$ y se
realizó a los puntos un ajuste mediante regresión lineal. Se analizó la ecuación obtenida
y se interpretó su pendiente. Luego, se comparó esta pendiente con el valor de la unidad
a través del porcentaje de error.
Por otra parte, a partir de los datos registrados, se realizó una gráfica de $2d$
(ordenadas) en función del tiempo (abcisas). Se realizó a los puntos un ajuste mediante
regresión lineal. Se analizó la ecuación obtenida y se interpretó su pendiente. Se
comparó mediante porcentaje de error la velocidad del sonido obtenida a partir de la
gráfica anterior con la velocidad calculada a partir de la expresión \eqref{eq:speed-sound}, en
donde $T_c$ es la temperatura en grados Celsius, registrada en el laboratorio.\par
\bigskip
Finalmente, a partir de los datos de la diferencia de tiempo $\Delta t$ (tiempo del pulso entre los
dos obstáculos) y usando la velocidad del sonido calculada anteriormente, se encontró la
distancia entre los obstáculos, $\Delta d$, y se comparó mediante porcentaje de error con las
distancias entre obstáculos registradas.
\section{Tratamiento de datos} \label{TD}
\section{Análisis de resultados}
\section{Conclusiones}
\section{Referencias} 
\bibliographystyle{unsrt}
\bibliography{../general-references/references}
\end{document}
