\documentclass[spanish,notitlepage,letterpaper, 12pt]{article} 
\usepackage[spanish]{babel} 
\usepackage{amsmath}
\usepackage{amsfonts}
\usepackage{amssymb}
\usepackage{graphicx}
\usepackage{geometry}      
\geometry{letterpaper}                  
\usepackage{epstopdf}
\usepackage{fancyhdr} % Paquete para encabezados y pies de pag
\usepackage{listings}
\usepackage{color}
\usepackage{placeins}
\usepackage{csquotes}
\usepackage{textcomp}
\usepackage{gensymb} 

\pagestyle{fancy} 
\chead{\bfseries Fase de formulación} 
\rhead{Grupo C1B. Subgrupo 2}
\lfoot{6 de Mayo de 2023}
\cfoot{Universidad Industrial de Santander} 
\rfoot{\thepage} 

\voffset = -0.25in 
\textheight = 8.0in 
\textwidth = 6.5in
\oddsidemargin = 0.in
\headheight = 20pt 
\headwidth = 6.5in
\renewcommand{\headrulewidth}{0.5pt}
\renewcommand{\footrulewidth}{0,5pt}
\begin{document}
\begin{titlepage}
    \begin{center}
        \includegraphics[width=0.4\textwidth]{../general-images/uis-logo.png}
        
        \vspace{0.5cm}
        \LARGE
        \textbf{Estudio de la amplitud de las oscilaciones armónicas amortiguadas y forzadas}
        
        \vspace{0.5cm}
        \large
        Informe 3
        
        \vfill
        
        \textbf{Daniel Esteban Vargas Reyes.} Estudiante - Geología\\
        \textbf{Nicolás Andrés Ramírez Calderón.} Estudiante - Ingeniería de Sistemas\\ 
        \textbf{Rodolfo Valentín Muñoz Vega.} Estudiante - Ingeniería Química\\

        \vspace{1.0cm}
        Presentado a la docente:
        
        \textbf{Zayda Paola Reyes Quijano}
        
        \vfill
        
        Escuela de Física - Física III\\
        Universidad Industrial de Santander\\
        Bucaramanga, Santander, Colombia\\
        05 de Mayo de 2023        
    \end{center}
\end{titlepage}

\tableofcontents

\newpage

\section{Formulación del problema}
En tiempos modernos es común vernos rodeados de una amplia variedad de dispositivos tecnológicos que utilizamos como herramientas para nuestro desarrollo personal y colectivo. Algunas de estas tecnologías funcionan como emisores/receptores de diferentes tipos de ondas, algunas mecánicas (Como
parlantes o micrófonos para sonido) y otras
electromagéticas (Como dispositivos de radiofrecuencias, WiFi, Bluetooth, etc).\par
\bigskip
Nuestro cuerpo no es ajeno a estos fenónemos físicos, ya que también interactuamos con los diferentes tipos de ondas; las cuerdas vocales y el oído tienen un rol en el sonido, mientras que algunas células de nuestra piel se ven claramente afectadas por las ondas electromagnéticas. \cite{IDEAM}\par
\bigskip
En la actualidad, entendemos que las ondas pueden provocar afectaciones en nuestra salud.  Por ejemplo, para ciertas frecuencias somos más sensibles a un alto nivel de intensidad de las ondas sonoras \cite{serway_jewett_2017}, al punto de que puede ser perjudicial para el oído someterse a ondas de
muy elevados decibelios. Por otra parte, las ondas electromagnéticas también pueden provocar efectos negativos, la radiación ultravioleta del sol o los rayos X pueden producir cáncer \cite{Wall2006-pi}.\par
\bigskip
De modo que, es pertinente preguntarnos: ¿Qué características tienen las ondas que pueden producir efectos nocivos en la salud?¿Qué dispotivos comunes pueden producir estas ondas y cuales no?¿Cuales serían las recomendaciones en relación con la exposición a dicho tipo de ondas?\par
\bigskip
Una hipótesis en esta situación podría sugerir que los dispositivos de uso doméstico no emiten ondas nocivas para la salud, ya que no existen impetuosas advertencias sobre el uso de estos dispositivos por parte de los fabricantes o una opinión consensuada por especialistas en salud. Mientras que
quizás algunos aparatos industriales o de uso médico podrían tener algún tipo de afectación en las células de nuestro cuerpo.
\section{Justificación}
Sin la información adecuada, carecemos de la capaidad de afirmar o negar la existencia de los efectos negativos en la salud producidos por las ondas electromagnéticas. Existen diversos estudios y teorías que sugieren que las ondas electromagnéticas pueden afectar la salud de diferentes maneras, como
causar alteraciones en el sueño, dolores de cabeza, fatiga, irritabilidad, problemas de memoria y concentración, entre otros. Además, se piensa que las ondas electromagnéticas pueden ser la causa de algunos problemas de salud más graves, como el cáncer, aunque esto todavía no se ha probado
definitivamente.\par
\bigskip
Por otra parte, algunos argumentan que la mayoría de los estudios epidemiológicos y de exposición a largo plazo siguen siendo inconsistentes y no proporcionan suficiente información para sacar conclusiones claras sobre la relación entre las ondas electromagnéticas y la salud humana. Aunque existen
algunas normativas para limitar la exposición de las personas a ciertos niveles de ondas electromagnéticas en algunos entornos, como en el trabajo o en los hogares,  no hay suficiente evidencia para modificar completamente la forma en que se utilizan las tecnologías inalámbricas hoy en día.\par
\bigskip
De modo que es pertinente conocer y comprobar las propiedades de las ondas a las que nos sometemos en el día a día para aportar a la construcción de una conclusión certera sobre el impacto de las ondas en nuestra salud.
\section{Objetivos}
\subsection{General}
Este proyecto tiene como objetivo determinar el impacto de los diferentes tipos de ondas (mecánicas y electromagnéticas) en nuestra cotidianidad por cuenta de los diferentes dispositivos tecnológicos así como por posibles causas externas a ellos.
\subsection{Específicos}
\begin{enumerate}
    \item Identificar las principales fuentes de ondas mecánicas y electromagnéticas, como teléfonos móviles, parlantes, ruido de ambiente, routers inalámbricos y electrodomésticos.
    \item Consultar las producciones académicas acerca de las repercusiones en la salud sobre la exposición a diferentes tipos de ondas.
    \item Medir las propiedades de las ondas que utilizan los dispositivos de uso cotidiano y compararlas con las especificaciones dadas por sus fabricantes.
    \item Proporcionar recomendaciones para la exposición regulada a los diferentes tipos de ondas a fin de garantizar la seguridad y la salud de las personas
\end{enumerate}
\section{Diseño metodológico}
El desarrollo del proyecto de investigación se puede secuenciar en tres fases principales:
\subsection{Primera fase}
En esta fase se busca ahondar en el entendimiento de los diferentes tipos de ondas y sus efectos en la salud, para ello se requiere de una profundización en los conceptos de ondas de sonido y ondas electromagnéticas.\par
\bigskip
Para esta fase también se pretende realizar una minuciosa consulta sobre el desarrollo académico en los efectos de las ondas en la salud, esto requiere indagar en las producciones academicas realizadas para entender la opinión general de los expertos en el tema.

\subsection{Segunda fase}
Esta fase pretende comprender el funcionamiento de los dispositivos tecnológicos de uso diario, de esta forma se pueden clasificar aquellos que hacen uso de ondas (y también que tipos de ondas utilizan) y descartar los que no hacen uso de ellas.\par
\bigskip
Posteriormente se hará necesario revisar las especificaciones técnicas de los dispositivos previamente mencionados, de esta manera podremos comprender las características de las ondas que son recibidas o emitidas por estos aparatos.
\subsection{Tercera fase}
Para comprobar la información obtenida en la fase anterior se hace necesario medir experimentalmente algunas de las caracterísiticas de las ondas que utilizan los dispositivos domésticos.\par
\bigskip
En esta fase se utilizarán instrumentos de medición que permitan percibir algunas propiedades de las ondas, tales como su amplitud, frecuencia, etc. La información será registrada para su posterior análisis.
\subsection{Cuarta fase}
Por último, se tomará la información obtenida en las fases anteriores para concluir si las ondas empleadas por los dispositivos de uso doméstico representan un riesgo para la salud y en caso de que no lo representen, bajo que circunstancias podrían hacerlo.
\section{Alcances}
Los alcances de este proyecto están presentados en tres ideas principales:
\begin{enumerate}
    \item Investigación y selección de frecuencias de sonido dañinas:\par
        Se llevará a cabo una investigación exhaustiva sobre las frecuencias de sonido que se ha demostrado que son peligrosas para el ser humano y puedan causarle daños. Con base en esta investigación, se seleccionarán dicha frecuencia para mantenerlas en estudio y desarrollar maneras de contrarrestar los daños que puedan causar.
    \item Evaluación de la efectividad del método para contrarrestar los daños:\par
        Se realizarán pruebas y evaluaciones para determinar la efectividad del sistema en la manera y la eficiencia de contrarrestar los efectos. Se utilizarán herramientas de seguimiento para poder monitorear los resultados y recopilarlos.
    \item Optimización del sistema o método encontrado para la ayuda contra los daños:
        Luego de recopilados los datos y analizada la eficiencia, se buscará un mejor funcionamiento, y cubrimiento, para poder realizar recomendaciones o posibles ayudas para los afectados por ondas dañinas. 
\end{enumerate}
\section{Bibliografía} 
\bibliographystyle{unsrt}
\bibliography{references}
\end{document}
