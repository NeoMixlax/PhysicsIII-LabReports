\documentclass[spanish,notitlepage,letterpaper, 12pt]{article} 
\usepackage[spanish]{babel} 
\usepackage{amsmath}
\usepackage{amsfonts}
\usepackage{amssymb}
\usepackage{graphicx}
\usepackage{geometry}      
\geometry{letterpaper}                  
\usepackage{epstopdf}
\usepackage{fancyhdr} % Paquete para encabezados y pies de pag
\usepackage{color}
\usepackage{placeins}
\usepackage{csquotes}
\usepackage{textcomp}
\usepackage{gensymb} 

\graphicspath{images/}

\pagestyle{fancy} 
\chead{\bfseries Informe X - Grupo C1B. Subgrupo 2} 
\rhead{date}
\cfoot{Universidad Industrial de Santander} 
\rfoot{\thepage} 

\voffset = -0.25in 
\textheight = 8.0in 
\textwidth = 6.5in
\oddsidemargin = 0.in
\headheight = 20pt 
\headwidth = 6.5in
\renewcommand{\headrulewidth}{0.5pt}
\renewcommand{\footrulewidth}{0,5pt}
\begin{document}
\begin{titlepage}
    \begin{center}
        \includegraphics[width=0.4\textwidth]{../general-images/uis-logo.png}
        
        \vspace{0.5cm}
        \LARGE
        \textbf{Estudio de la amplitud de las oscilaciones armónicas amortiguadas y forzadas}
        
        \vspace{0.5cm}
        \large
        Informe 3
        
        \vfill
        
        \textbf{Daniel Esteban Vargas Reyes.} Estudiante - Geología\\
        \textbf{Nicolás Andrés Ramírez Calderón.} Estudiante - Ingeniería de Sistemas\\ 
        \textbf{Rodolfo Valentín Muñoz Vega.} Estudiante - Ingeniería Química\\

        \vspace{1.0cm}
        Presentado a la docente:
        
        \textbf{Zayda Paola Reyes Quijano}
        
        \vfill
        
        Escuela de Física - Física III\\
        Universidad Industrial de Santander\\
        Bucaramanga, Santander, Colombia\\
        05 de Mayo de 2023        
    \end{center}
\end{titlepage}

\tableofcontents

\newpage

\section{Resumen}

\section{Introducción}

\subsection{Marco teórico} \label{I.MT}
\section{Metodología}
\subsection{Materiales}
\subsection{Fase 1}
\subsection{Fase 2}
\subsection{Fase 3}
\section{Tratamiento de datos} \label{TD}
\section{Análisis de resultados}
\section{Conclusiones}
\section{Referencias} 
\bibliographystyle{unsrt}
\bibliography{../general-references/references}
\end{document}
